\documentclass[12pt]{elsarticle}
\usepackage[bottom=2cm,top=3cm,left=3cm,right=2cm]{geometry}
\usepackage{url}
\makeatletter
\def\ps@pprintTitle{%
      	\let\@oddhead\@empty
      	\let\@evenhead\@empty
      	\def\@oddfoot{\reset@font\hfil\thepage\hfil}
      	\let\@evenfoot\@oddfoot
}
\makeatother

\usepackage{babelbib}

\usepackage[brazilian]{babel} % Traduz alguns termos para o português
\usepackage[utf8]{inputenc} % Reconhece acentuação
\usepackage{setspace}

\onehalfspacing

\begin{document}

	\begin{frontmatter}

		\title{Teoria da Decisão\\ \resizebox{15cm}{0.3cm}{Métodos Escalares de Otimização Vetorial e Tomada de Decisão Assistida}}
		\author{Rafael Carneiro de Castro - 2013030210\\
			Davi Pinheiro Viana - 2013029912}
		\address{Minas Gerais, Brasil}
		
	\end{frontmatter}
	
	\section{Introdução:}
	O presente trabalho tem o objetivo de resolver um problema de otimização utilizando as técnicas escalares e de decisão assistida estudados em sala de aula, colocando em prática grande parte dos conceitos da matéria.
	
	O problema a ser resolvido é o seguinte:
	\textit{Uma empresa possui um conjunto de M máquinas que devem ser utilizadas para processar N tarefas indivisíveis. Cada máquina $i$ leva um tempo $t_{ij}$ para processar uma tarefa $j$ e pode processar uma única tarefa por vez. Todas as tarefas possuem uma mesma data ideal de entrega $d$, sendo que cada tarefa $j$ sofre uma penalidade $w_j$ proporcional a cada dia que ela é entregue adiantada ou atrasada em relação a $d$.}
	
	\section{Formulação do Problema:}
	Em primeiro momento é preciso construir uma função objetivo e suas eventuais restrições para minimização do tempo total de entrega de todas as tarefas. Considere $C_i$ como sendo o tempo necessário para se terminar uma tarefa, executada pela máquina $i$. Assim:
	 \[C_i = \sum_{j=1}^{N}t_{ij}*x_{ij}\ \forall\ i \in\ (1,...,M) \]
	O objetivo então se torna:
	\[min\ C_{max} \]
	\[C_{max} = max(C_i)\ \forall\ i \in\ (1,...,M) \]
	sujeito a:
	\[\sum_{i=1}^{M}x_{ij}=1\ \forall\ j \in\ (1,...,N) \]
	\[\sum_{j=1}^{N}t_{ij}*x_{ij} <= C_{max}\ \forall\ i \in\ (1,...,M) \]
	\[X_{ij} \in (0, 1) \]
	Com estas restrições, garante-se que cada tarefa vai ser cumprida por uma única máquina.
	
%%	\begin{figure}[h]
%%		\centering
%%		\includegraphics[width=17cm]{img/sinal.png}
%%		\caption{Sinal senoidal com frequência de 480 Hz.}
%%		\label{fig:sinal}
%%	\end{figure}
	


\end{document}