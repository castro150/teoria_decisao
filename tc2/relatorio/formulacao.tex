A formulação do problema foi dividida em duas partes, como é discutido a seguir:

\subsubsection{Minimização do custo de manutenção total} Em primeiro
momento, é preciso construir uma função objetivo e suas eventuais restrições para minimização do custo de manutenção total. Considerando $C_{m_i}(x_i)$ como o custo de manutenção do equipamento $i$ em função do plano de manutenção $x_i$, têm-se a seguinte formulação:

\begin{equation}
\mathrm{min}\ \sum_{i=1,}^{n} C_{m_i} (x_i) 
\label{sum_cm}
\end{equation}

sujeito a:
\begin{equation}
x_i \in \mathcal{X}\ \forall i\ \in 1, ..., n
\label{rest1_cm}
\end{equation}
\begin{equation}
C_{m_i} \in \mathcal{C}_{m}\ \forall i\ \in 1, ..., n
\label{rest2_cm}
\end{equation}
Em que $n$ é o número de equipamentos que, no caso do problema a ser resolvido, é igual a 500. A equação \ref{sum_cm} representa o custo de manutenção total que é o somatório dos custos de manutenção de cada equipamento $i$. A restrição \ref{rest1_cm} indica que cada equipamento $i$ pode ter um plano de manutenção $x_i$ que esteja dentro do conjunto $\mathcal{X}$ de planos pré-definidos, no caso do problema, $\mathcal{X} = \{1,2,3\}$. A restrição \ref{rest2_cm} indica que o custo de manutenção de cada equipamento também deve estar dentro de um conjunto pré-definido $\mathcal{C}_{m}$, sendo que o valor depende do plano de manutenção.

\subsubsection{Minimização do custo esperado de falha total} Agora, uma função objetivo para tratar a minimização do custo esperado de falha total é formulada. Considerando $C_{f_i}(x_i)$ como o custo de falha do equipamento $i$ em função do plano de manutenção $x_i$, têm-se a seguinte formulação:

\begin{equation}
C_{f_i} = p_{i,x_i} \cdot c_{f_i}
\label{cf}
\end{equation}

Onde $p_{i,x_i}$ é a probabilidade de falha de um equipamento $i$, sob o plano de manutenção $x_i$, até um dado horizonte de planejamento da manutenção $\Delta t$. Ela é estimada pela equação \ref{pf} que determina a probabilidade de falha de um equipamento até $\Delta t$ dado que ele não falhou até a data atual ($t_0$). No caso do problema, será utilizado $\Delta t = 5$ anos.

\begin{equation}
p_{i,x_i} = \frac{F_i(t_0 + x_i\Delta t) - F_i(t_0)}{1 - F_i(t_0)}
\label{pf}
\end{equation}

Em que:

\begin{equation}
F_i(t) = 1 - \exp{\left[ -{ \left( \frac{t}{\eta_i} \right) }^{\beta_i} \right]}
\end{equation}

Os parâmetros $\eta$, $\beta$ dependem também do plano de manutenção $i$ e são dados. Com isso, têm-se o seguinte modelo:

\begin{equation}
\mathrm{min}\ \sum_{i=1,}^{n} C_{f_i} (x_i) 
\label{sum_cf}
\end{equation}

sujeito a:
\begin{equation}
x_i \in \mathcal{X}\ \forall i\ \in 1, ..., n
\label{rest1_cf}
\end{equation}
\begin{equation}
c_{f_i} \in \mathcal{C}_{f}\ \forall i\ \in 1, ..., n
\label{rest2_cf}
\end{equation}
\begin{equation}
\beta_{i} \in \mathcal{B}\ \forall i\ \in 1, ..., n
\label{rest3_cf}
\end{equation}
\begin{equation}
\eta_{i} \in \mathcal{N}\ \forall i\ \in 1, ..., n
\label{rest4_cf}
\end{equation}

Em que $n$ é o número de equipamentos que, no caso do problema a ser resolvido, é igual a 500. A equação \ref{sum_cf} representa o custo esperado de falha total que é o somatório dos custos esperados de falha de cada equipamento $i$. A restrição \ref{rest1_cf} indica que cada equipamento $i$ pode ter um plano de manutenção $x_i$ que esteja dentro do conjunto $\mathcal{X}$ de planos pré-definidos, no caso do problema, $\mathcal{X} = \{1,2,3\}$. As restrições \ref{rest2_cf}, \ref{rest3_cf} e \ref{rest4_cf} indicam, respectivamente que $c_{f_i}$, $\beta_i$, $\eta_i$ devem estar dentro de conjuntos pré-definidos, sendo que o valor depende do plano de manutenção.

\subsubsection{Minimização de ambos os custos} O problema a ser resolvido envolve a minimização do custo de manutenção total \emph{e também} do custo de falha total, logo, é necessária a formulação de um problema biobjetivo para o problema. Para a formulação, foi escolhido o método \emph{Soma Ponderada}. Nele, a função biobjetivo é formada por uma soma das funções objetivos anteriores, sendo cada uma multiplicada por um peso. A variação desses pesos é que faz com que a fronteira Pareto seja formada. Esse método foi escolhido por ser de fácil implementação. Com isso, a formulação do problema biobjetivo é a seguinte:

\begin{equation}
\mathrm{min}\ w_1 \cdot \sum_{i=1,}^{n} C_{m_i} (x_i) + w_2 \cdot \sum_{i=1,}^{n} C_{f_i} (x_i) 
\label{sum_biobj}
\end{equation}

sujeito a:
\begin{equation}
x_i \in \mathcal{X}\ \forall i\ \in 1, ..., n
\label{rest1_biobj}
\end{equation}
\begin{equation}
c_{f_i} \in \mathcal{C}_{f}\ \forall i\ \in 1, ..., n
\label{rest2_biobj}
\end{equation}
\begin{equation}
\beta_{i} \in \mathcal{B}\ \forall i\ \in 1, ..., n
\label{rest3_biobj}
\end{equation}
\begin{equation}
\eta_{i} \in \mathcal{N}\ \forall i\ \in 1, ..., n
\label{rest4_biobj}
\end{equation}