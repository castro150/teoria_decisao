Conforme estudado em sala de aula, o método Promethee II se baseia na comparação da avaliação de alternativas nos critérios tomando como base uma função de generalização de critérios. No caso deste trabalho, escolheu-se utilizar o Critério Usual, uma função que retorna 1 caso $c_j(a_i) - c_j(a_k) > 0$, ou seja, caso a avaliação no critério $c_j$ de $a_i$ seja melhor que a avaliação $a_k$ no critérios $c_j$, por ser uma generalização de critérios mais intuitiva. Como no problema em estudo os critérios de custos são melhores quanto menor for o valor, apenas precisamos adaptar o Critério Usual para retornar 1 quando $custo(a_k) - custo(a_i) > 0$. Assim, cria-se as matrizes $P_j$ com a comparação par a par de todas as alternativas da fronteira Pareto em cada um dos dois critérios, preenchendo com os resultados da generalização de critérios de cada par. Agora, a matriz $P$, que será útil para os cálculos de fluxos de preferências entre as alternativas, é calculada a partir da relação:
\begin{equation}
P(a_i, a_k) = \frac{\sum_{j=1}^{2} w_j \cdot P_j(a_i, a_k)}{\sum_{j=1}^{2} w_j}
\label{promethee}
\end{equation}

Muito se discute na literatura sobre formas nebulosas de se implementar o Promethee II. Uma das possíveis abordagens, e que se encaixa no problema em questão, é utilizar os pesos dos critérios como números nebulosos, representando a imprecisão do decisor em escolher pesos para cada critério. Assim, os valores de $w_i$ no somatória do denominador da relação são números nebulosos. 

Por fim, o fluxo de preferência em cada alternativa é calculado pela subtração entre o fluxo que entra no nó de uma alternativa e o fluxo que sai do nó dessa alternativa. Uma alternativa sobreclassifica outra se o seu fluxo de preferência for maior. Elas são indiferentes entre si se possuem o mesmo fluxo. Toda esta lógica do Promethee II fuzzy pode ser vista no arquivo \texttt{FPrometheeII.m}. O arquivo cria uma matriz de sobreclassificação, que vai ter o número 1 em uma célula caso a alternativa daquela linha sobreclassifique a alternativa da coluna, 0 caso sejam indiferentes, e -1 caso a alternativa da coluna sobreclassifique a da linha. As alternativas que mais sobreclassificam outras alternativas são tidas então como as melhores opções. No problema estudado, segundo o algoritmo do Promethee II fuzzy, as melhores alternativas possuem custo de manutenção total igual a \textit{1000} e custo esperado de falha total igual a \textit{1048.2}, considerando os respectivos pesos igual a \textit{0.7} e \textit{0.3}.