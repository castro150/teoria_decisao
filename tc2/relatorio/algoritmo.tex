Nesta seção serão discutidos e exibidos os algoritmos para solução do problema multiobjetivo.

Olhando para a equação \ref{sum_biobj} é possível perceber que, minimizando o custo de cada equipamento, minimiza-se também o somatório dos custos. Assim, para resolução do problema biobjetivo foi utilizada uma estratégia gulosa. Nela, para cada equipamento, é feito um teste com cada um dos planos de manutenção e é escolhido aquele que gera menor custo. Têm-se então, um algoritmo cuja complexidade é O($n\cdot m$) em que $n$ é o número de equipamentos e $m$ é o número de planos de manutenção. No caso do problema a ser resolvido no trabalho, para cada par de pesos escolhido (encontrar solução da fronteira Pareto), são feitas 1500 avaliações da função objetivo. Segue, abaixo, um pseudocódigo do funcionamento do algoritmo:

\begin{algorithm}
	\caption{Estratégia gulosa}
	\begin{algorithmic}[1]
		\For{$i = 1$ to ${n}$}
			\State $cBest = w_1 \cdot c_m(\mathcal{X}_1) + w_2 \cdot c_f(\mathcal{X}_1)$
			\State $x_i = \mathcal{X}_1$
			\For{$j = 2$ to ${m}$}
					\If {$(w_1 \cdot c_m(\mathcal{X}_j) + w_2 \cdot c_f(\mathcal{X}_j)) < cBest$}
					\State $cBest = w_1 \cdot c_m(\mathcal{X}_j) + w_2 \cdot c_f(\mathcal{X}_j)$
					\State $x_i = \mathcal{X}_j$
					\EndIf
			\EndFor
		\EndFor
		
	\end{algorithmic}
\end{algorithm}

Essa estratégia foi escolhida por ser simples de implementar e por retornar uma solução exata para o problema. Além disso, é uma solução relativamente barata computacionalmente e que retorna o resultado rapidamente.

O algoritmo que utiliza a estratégia gulosa para resolver a função objetivo pode ser encontrado no arquivo \texttt{Guloso.m} e o algoritmo que implementa a \emph{Soma Ponderada} variando os pesos da função objetivo pode ser encontrado no arquivo \texttt{SomaPonderada.m}, ambos no mesmo diretório deste relatório.