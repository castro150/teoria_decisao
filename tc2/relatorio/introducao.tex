O presente trabalho tem o objetivo de resolver um problema de otimização multiobjetivo e, utilizando técnicas escalares de decisão assistida estudadas em sala de aula, encontrar a melhor solução para este problema, colocando em prática grande parte dos conceitos da matéria.

O problema a ser resolvido é o seguinte: \emph{Deseja-se determinar a política de manutenção ótima para cada um dos 500 equipamentos de	uma empresa, considerando-se a minimização do custo de manutenção e a minimização do custo de falha esperado.}

No problema, o custo de manutenção total é a soma dos custos dos planos de manutenção adotados para todos os equipamentos. Sendo que, o valor do custo de cada plano de manutenção é dado. O custo esperado de falha de cada equipamento $i$, sob o plano de manutenção $j$, é o produto da probabilidade de falha ($p_{i,j}$) e o custo de falha do equipamento (este último é dado). O
custo esperado de falha total é a soma dos custos esperados de falha de todos os equipamentos.

Deve ser feita a formulação e resolução do problema multiobjetivo e o resultado encontrado deve ser avaliado baseado no indicador de qualidade hipervolume (s-metric). Esse indicador é utilizado para mensurar as propriedades de convergência e diversidade da fronteira Pareto "aproximada" obtida.

Além disso, deve ser aplicada também a utilização de técnicas de análise de decisão ELECTRE II, PROMETHEE II \emph{fuzzy} e AHP para decidir qual a melhor solução dentre as encontradas para o problema.