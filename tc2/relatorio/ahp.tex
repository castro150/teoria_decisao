O método AHP utilizado aqui é o mesmo que o estudado em sala de aula. Para tanto, escolheu-se 5 soluções da fronteira de Pareto encontrada na otimização biobjetivo. Chamaremos aqui de \textit{c1} (critério 1) o custo de manutenção total, e de \textit{c2} (critério 2) o custo esperado de falha total. Os valores destes critérios para as 5 soluções escolhidas podem ser vistos na tabela a seguir.
\begin{table}[h]
	\centering
	\begin{tabular}{ | l | l | l | l |}
		\hline
		Alternativa & c1 & c2 \\ \hline
		a1 & 1000 & 1048,2 \\ \hline
		a2 & 622 & 1184,3 \\ \hline
		a3 & 396 & 1340,9 \\ \hline
		a4 & 40 & 1695,3 \\ \hline
		a5 & 0 & 1745,5 \\ \hline
	\end{tabular}
	\label{table:c-ahp}
	\caption{Avaliação das alternativas nos critérios.}
\end{table}

Agora, para a definição das prioridades de cada alternativa em cada critério, faz-se as tabelas de prioridade dando notas às alternativas. Para o primeiro critério, a tabela construída é:
\begin{table}[h]
	\centering
	\begin{tabular}{ | l | l | l | l | l | l | l | }
		\hline
		c1 & a1 & a2 & a3 & a4 & a5 & Prioridades \\ \hline
		a1 & 1 & 0.333 & 0.2 & 0.143 & 0.111 & 0.0348 \\ \hline
		a2 & 3 & 1 & 0.333 & 0.2 & 0.143 & 0.0678 \\ \hline
		a3 & 5 & 3 & 1 & 0.333 & 0.2 & 0.1343 \\ \hline
		a4 & 7 & 5 & 3 & 1 & 0.333 & 0.2602 \\ \hline
		a4 & 9 & 7 & 5 & 3 & 1 & 0.5028 \\ \hline
	\end{tabular}
	\label{table:c1}
	\caption{Prioridades critério 1.}
\end{table}

Para o segundo critério, a tabela construída é:
\begin{table}[h]
	\centering
	\begin{tabular}{ | l | l | l | l | l | l | l | }
		\hline
		c1 & a1 & a2 & a3 & a4 & a5 & Prioridades \\ \hline
		a1 & 1 & 3 & 5 & 8 & 9 & 0.5029 \\ \hline
		a2 & 0.333 & 1 & 3 & 6 & 7 & 0.2623 \\ \hline
		a3 & 0.2 & 0.333 & 1 & 4 & 5 & 0.1395 \\ \hline
		a4 & 0.125 & 0.167 & 0.25 & 1 & 3 & 0.0610 \\ \hline
		a5 & 0.111 & 0.143 & 0.2 & 0.333 & 1 & 0.0344 \\ \hline
	\end{tabular}
	\label{table:c2}
	\caption{Prioridades critério 2.}
\end{table}

Como visto em sala de aula, as prioridades são calculadas a partir da normalização dos termos nas colunas, tirando a média de cada linha. Agora escolhendo o peso do custo de manutenção total como sendo 0.4 e o peso do custo esperado de falha total como sendo 0.6 (consideramos que o custo de falha tem maior impacto), para cada alternativa basta multiplicar pelos pesos dos critérios cada uma de suas prioridades e somar, comparando assim o resultado obtido para todas:
\begin{equation}
p_i = \sum_{j=1}^{2} w_j*P_{ij}
\label{sum_cm}
\end{equation}

O código do arquivo \texttt{AHP.m} lê as matrizes de prioridade dos arquivos \textit{AHPcriterio1.csv} e \textit{AHPcriterio2.csv}, faz os cálculos de prioridade apresentados e calcula o somatório para as comparações finais. Este somatório para cada alternativa pode ser visto na tabela seguinte:
\begin{table}[h]
	\centering
	\begin{tabular}{ | l | l | }
		\hline
		Alternativa & Prioridade Final \\ \hline
		a1 & 0.3156 \\ \hline
		a2 & 0.1845 \\ \hline
		a3 & 0.1374 \\ \hline
		a4 & 0.1407 \\ \hline
		a5 & 0.2218 \\ \hline
	\end{tabular}
	\label{table:ahp-result}
	\caption{Prioridades critério 2.}
\end{table}

Como se pode notar, para os pesos escolhidos para os dois critérios, a alternativa 1 se mostrou a mais promissora, mas a alternativa 5 também está próxima desta. Um ajuste dos pesos poderia trazer um resultado final diferente.